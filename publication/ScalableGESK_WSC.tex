% Autogenerated translation of ScalableGESK_WSC.md by Texpad
% To stop this file being overwritten during the typeset process, please move or remove this header

\documentclass[12pt]{book}
\usepackage{graphicx}
\usepackage[utf8]{inputenc}
\usepackage[a4paper,left=.5in,right=.5in,top=.3in,bottom=0.3in]{geometry}
\setlength\parindent{0pt}
\setlength{\parskip}{\baselineskip}
\renewcommand*\familydefault{\sfdefault}
\usepackage{hyperref}
\pagestyle{plain}
\begin{document}
\Large

+++
title = "A Scalable Approach to Enhancing Stochastic Kriging with Gradients"
date = 2018-04-29T18:06:44+08:00
draft = false

\chapter*{Authors. Comma separated list, e.g. \texttt{["Bob Smith", "David Jones"]}.}

authors = ["Haojun Huo", "Xiaowei Zhang", "Zeyu Zheng"]

\chapter*{Publication type.}

\chapter*{Legend:}

\chapter*{0 = Uncategorized}

\chapter*{1 = Conference paper}

\chapter*{2 = Journal article}

\chapter*{3 = Manuscript}

\chapter*{4 = Report}

\chapter*{5 = Book}

\chapter*{6 = Book section}

\chapter*{7 = Working paper}

publication\_types = ["1"]

\chapter*{Publication name and optional abbreviated version.}

publication = "WSC18"
publication\_short = ""

\chapter*{Abstract and optional shortened version.}

abstract = "It is known that incorporating gradient information can significantly enhance the prediction accuracy of stochastic kriging. However, such an enhancement cannot be scaled trivially to high-dimensional design space, since one needs to invert a covariance matrix of size \$n(d+1)times n(d+1)\$ that captures the spatial correlations between the responses and the gradient estimates at the design points, where \$d\$ and \$n\$ are the dimensionality and the number of design points, respectively. Not only is the inversion computationally inefficient, but also numerically unstable since the covariance matrix is often ill-conditioned. We address the scalability issue via a novel approach without resorting to matrix approximations. By virtue of the so-called Markovian covariance functions, the associated covariance matrix can be invertible analytically, thereby improving both the efficiency and stability dramatically. Numerical experiments demonstrate that the proposed approach can handle large-scale problems where prior methods fail completely.  "

abstract\_short = " "

\chapter*{Featured image thumbnail (optional)}

image\_preview = ""

\chapter*{Is this a selected publication? (true/false)}

selected = false

\chapter*{Projects (optional).}

\chapter*{Associate this publication with one or more of your projects.}

\chapter*{Simply enter the filename (excluding '.md') of your project file in \texttt{content/project/}.}

\chapter*{E.g. \texttt{projects = ["deep-learning"]} references \texttt{content/project/deep-learning.md}.}

projects = ["kriging"]

\chapter*{Tags (optional).}

\chapter*{Set \texttt{tags = []} for no tags, or use the form \texttt{tags = ["A Tag", "Another Tag"]} for one or more tags.}

tags = ["Stochastic Kriging", "Gaussian Process", "Big Data"]

\chapter*{Links (optional).}

url\emph{pdf = "pdf/ScalableGESK-WSC18-online.pdf"
url}preprint = ""
url\emph{code = ""
url}dataset = ""
url\emph{project = ""
url}slides = ""
url\emph{video = ""
url}poster = ""
url\_source = ""

\chapter*{Custom links (optional).}

\chapter*{Uncomment line below to enable. For multiple links, use the form \texttt{[\{...\}, \{...\}, \{...\}]}.}

\chapter*{url\_custom = [\{name = "Custom Link", url = "http://example.org"\}]}

\chapter*{Does this page contain LaTeX math? (true/false)}

math = true

\chapter*{Does this page require source code highlighting? (true/false)}

highlight = true

\chapter*{Featured image}

\chapter*{Place your image in the \texttt{static/img/} folder and reference its filename below, e.g. \texttt{image = "example.jpg"}.}

[header]
image = ""
caption = ""

+++

\end{document}
